% Format teze zasnovan je na paketu memoir
% http://tug.ctan.org/macros/latex/contrib/memoir/memman.pdf ili
% http://texdoc.net/texmf-dist/doc/latex/memoir/memman.pdf
% 
% Prilikom zadavanja klase memoir, navedenim opcijama se podešava 
% veličina slova (12pt) i jednostrano štampanje (oneside).
% Ove parametre možete menjati samo ako pravite nezvanične verzije
% mastera za privatnu upotrebu (na primer, u b5 varijanti ima smisla 
% smanjiti 
\documentclass[12pt,oneside]{memoir} 

% Paket koji definiše sve specifičnosti master rada Matematičkog fakulteta
\usepackage[latinica]{matfmaster} 

\usepackage[bottom]{footmisc}
\usepackage{latexsym}
\usepackage{amssymb}
\usepackage{amsmath}
%
% Podrazumevano pismo je ćirilica.
%   Ako koristite pdflatex, a ne xetex, sav latinički tekst na srpskom jeziku
%   treba biti okružen sa \lat{...} ili \begin{latinica}...\end{latinica}.
%
% Opicija [latinica]:
%   ako želite da pišete latiniciom, dodajte opciju "latinica" tj.
%   prethodni paket uključite pomoću: \usepackage[latinica]{matfmaster}.
%   Ako koristite pdflatex, a ne xetex, sav ćirilički tekst treba biti
%   okružen sa \cir{...} ili \begin{cirilica}...\end{cirilica}.
%
% Opcija [biblatex]:
%   ako želite da koristite reference na više jezika i umesto paketa
%   bibtex da koristite BibLaTeX/Biber, dodajte opciju "biblatex" tj.
%   prethodni paket uključite pomoću: \usepackage[biblatex]{matfmaster}
%
% Opcija [b5paper]:
%   ako želite da napravite verziju teze u manjem (b5) formatu, navedite
%   opciju "b5paper", tj. prethodni paket uključite pomoću: 
%   \usepackage[b5paper]{matfmaster}. Tada ima smisla razmisliti o promeni
%   veličine slova (izmenom opcije 12pt na 11pt u \documentclass{memoir}).
%
% Naravno, opcije je moguće kombinovati.
% Npr. \usepackage[b5paper,biblatex]{matfmaster}

% Pomoćni paket koji generiše nasumičan tekst u kojem se javljaju sva slova
% azbuke (nema potrebe koristiti ovo u pravim disertacijama)
\usepackage[latinica]{pangrami}

% Datoteka sa literaturom u BibTex tj. BibLaTeX/Biber formatu
\bib{matfmaster-primer}

% Ime kandidata na srpskom jeziku (u odabranom pismu)
\autor{Miloš P. Miković}
% Naslov teze na srpskom jeziku (u odabranom pismu)
\naslov{Algoritmi za rešavanje problema najkraće zajedničke nadniske}
% Godina u kojoj je teza predana komisiji
\godina{2021}
% Ime i afilijacija mentora (u odabranom pismu)
\mentor{dr Aleksandar \textsc{Kartelj}, docent\\ Univerzitet u Beogradu, Matematički fakultet}
% Ime i afilijacija prvog člana komisije (u odabranom pismu)
\komisijaA{dr Vladimir \textsc{Filipović}, redovni profesor\\ Univerzitet u Beogradu, Matematički fakultet}
% Ime i afilijacija drugog člana komisije (u odabranom pismu)
\komisijaB{dr Stefan \textsc{Mišković}, docent\\ Univerzitet u Beogradu, Matematički fakultet}
% Ime i afilijacija trećeg člana komisije (opciono)
% \komisijaC{}
% Ime i afilijacija četvrtog člana komisije (opciono)
% \komisijaD{}
% Datum odbrane (odkomentarisati narednu liniju i upisati datum odbrane ako je poznat)
% \datumodbrane{}

% Apstrakt na srpskom jeziku (u odabranom pismu)
\apstr{%
% \pangrami
}

% Ključne reči na srpskom jeziku (u odabranom pismu)
\kljucnereci{optimizacija, pretraga bima, analiza sekvenci}

\begin{document}
% ==============================================================================
% Uvodni deo teze
\frontmatter
% ==============================================================================
% Naslovna strana
\naslovna
% Strana sa podacima o mentoru i članovima komisije
\komisija
% Strana sa posvetom (u odabranom pismu)
\posveta{Hvala profesoru Aleksandru Kartelju.}
% Strana sa podacima o disertaciji na srpskom jeziku
\apstrakt
% Sadržaj teze
\tableofcontents*

% ==============================================================================
% Glavni deo teze
\mainmatter
% ==============================================================================

% ------------------------------------------------------------------------------
\chapter{Uvod}
% ------------------------------------------------------------------------------
% \pangrami
Problem najkraće zajedničke nadniske (\textit{eng.} Shortest Common Supersequence Problem)
jedan je od dobro poznatih NP-teških problema optimizacije u oblasti analize reči \cite{ProbabilisticBS}.
Ukratko, PNZN\footnote{U nastavku teksta PNZN ćemo koristiti kao skraćenicu za problem najkraće zajedničke nadniske}
se može opisati kao problem pronalaženja najkraće sekvence $\mathcal{S}$ sačinjene
od simbola zadate konačne Azbuke $\Sigma$, tako da su sve sekvence iz unapred zadatog konačnog skupa
$\mathcal{L}$ sadržane u sekvenci $\mathcal{S}$. Kada se kaže da su sve reči iz skupa $\mathcal{L}$
sadržane, misli se na to da se svaka reč iz skupa $\mathcal{L}$ može dobiti uklanjanjem simbola iz reči $\mathcal{S}$ ali 
u zadatom redosledu \cite{SCSSProblemDef}. Opisi u kojim se sve oblastima koristi SCSS...dasdasdasdasdasdas
dasdsadsad sadas sad sadsa dasdsadsadsadsadsadas sa dsad asd sad asd asd as dasdasdasd sadasds

\section{Problem najkraće zajedničke nadniske}
U ovom poglavlju formalno ćemo definisati PNZN, ali pre toga uvešćemo potrebnu notaciju koja će biti korišćena u nastavku
teksta. Konačna azbuku sastoji se od konačnog broja slova i označavaćemo je sa $\Sigma$. Svaka konačna reč
$\omega=\omega(1)\omega(2)...\omega(n)$ sastoji se od konačnog broja slova azbuke gde $\omega(j)\in\Sigma$ predstavlja j-to slovo reči $\omega\in\Sigma^*$.
Duzinu reči $\omega$ označavaćemo sa $|\omega|$, praznu reč sa $\varepsilon$ i važi da $|\varepsilon|=0$. U skladu sa uvedenom
notacijom $|\Sigma|$ predstavlja kardinalnost azbuke. Sa $\omega\unrhd\alpha$ označavaćemo broj pojavljivanja slova $\alpha$
u reči $\omega$ ($\omega(1)\omega(2)...\omega(n)\unrhd\alpha=\sum_{1<=i<=n,\omega(i)=\alpha}1$). Reč koja se dobija dodavanjem
slova $\alpha$ na početak reči $\omega$ označavaćemo sa $\alpha\omega$ (takođe ćemo pisati $\omega=\alpha\omega^{'}$), slično reč koja se dobija skidanjem slova $\alpha$ sa početka
reči $\omega$ sa $\omega|_{\alpha}$. Brisanje slova $\alpha$ sa početka svake reči u zadatom skupu, u skladu sa uvedenom notacijom
definišemo kao $\{\omega_{1},\omega_{2},...,\omega_{n}\}|_{\alpha}=\{\omega_{1}|_{\alpha},\omega_{2}|_{\alpha},...,\omega_{n}|_{\alpha}\}$.

Neka važi da $\omega_{1},\omega_{2}\in\Sigma^*$, za reč $\omega_{1}$ kažemo da je
supersekvenca reči $\omega_{2}$ u oznaci $\omega_{1}\succ\omega_{2}$ ako važi sledeća rekurzivna definicija \cite{ProbabilisticBS}:
\\
\begin{equation}
\begin{aligned}
\omega_{1}\succ\varepsilon &\triangleq \textrm{Tačno}\\
\varepsilon\succ\omega_{2} &\triangleq \textrm{Netačno, \:\:Ako } \omega_{2}\neq\varepsilon\\
\alpha\omega_{1}\succ\alpha\omega_{2} &\triangleq \omega_{1}\succ\omega_{2}\\
\alpha\omega_{1}\succ\beta\omega_{2} &\triangleq \omega_{1}\succ\beta\omega_{2} \textrm{, \:\:Ako } \alpha\neq\beta
\end{aligned}
\end{equation}
\\

Zapravo, $\omega_{1}\succ\omega_{2}$ označava da se svi simboli iz $\omega_{2}$ nalaze u $\omega_{1}$ u datom redosledu,
ali ne nužno uzastopno. Na primer, za datu azbuku $\Sigma=\{a,c,t,g\}$, važi $agcatg \succ act$.
Sada možemo formalno definisati PNZN.

% % Primeri citiranja\mathsf{Y} 
% Ovo je rečenica u kojoj se javlja citat \cite{PetrovicMikic2015}.
% Još jedan citat \cite{GuSh:243}.
% % Primeri navodnika
% Isprobavamo navodnike: "Rekao je da mu se javimo sutra".
% % Primer referisanja na tabelu (koja se javlja kasnije)
% U tabeli \ref{tbl:rezultati} koja sledi prikazani su rezultati eksperimenta.
% % Primer kraćeg ćiriličkog teksta
% {\cir Ово је пример ћириличког текста који се јавља у латиничком документу.}
% U ovoj rečenici se javlja jedna reč na {\cir ћирилици}.
% % Primer korišćenja fusnota
% % Iza ove rečenice sledi fusnota.\footnote{Ovo je fusnota.}

% % Primer dužeg ćirličkog teksta
% \begin{cirilica}
%   Ово је мало дужи блок текста исписан ћириличким писмом у оквиру
%   латиничког документа. Фијуче ветар у шибљу, леди пасаже и куће иза
%   њих и гунђа у оџацима.
% \end{cirilica}

% % Primer korišćenja tabele
% \begin{table}
% \centering
% \caption{Rezultati}
% \label{tbl:rezultati}
% \begin{tabular}{c>{\centering}p{2cm}c}
% \toprule
% 1 & 2 & 3\\\midrule
% 4 & 5 & 6\\\cmidrule(rl){1-2}
% 7 & 8 & 8\\
% \bottomrule
% \end{tabular}
% \end{table}

% % Primer korišćenja slike
% \begin{figure}[!ht]
%   \centering
%   \label{fig:grafikon}
%   \includegraphics[width=0.5\textwidth]{graph.png}
%   \caption{Grafikon}
% \end{figure}


% % Primer jednostavnije matematičke formule
% Evo i jedan primer matematičke formule: $e^{i\pi} + 1 = 0$. 
% % Primer referisanja na sliku
% Na slici \ref{fig:grafikon} prikazan je jedan grafikon.

% % primer kompleksnije matematičke formule
% $$
% \int_a^b f(x)\ \mathrm{d}x \ =_{def}\ \lim_{\max{\Delta x_k \rightarrow 0}} \sum_{k=1}^n f(x_k^*)\Delta x_k
% $$

% % primer referisanja na poglavlja i strane poglavlja
% Više detalja biće dato u glavi \ref{chp:razrada} na strani \pageref{chp:razrada}.

% % primer liste
% Možemo praviti i nabrajanja:
% \begin{enumerate}
% \item Analiza 1
% \item Linearna algebra
% \item Analitička geometrija
% \item Osnovi programiranja
% \end{enumerate}

% \pangrami

% ------------------------------------------------------------------------------
\chapter{Razrada}
\label{chp:razrada}

% ------------------------------------------------------------------------------

% \pangrami

% \pangrami

% ------------------------------------------------------------------------------
\chapter{Zaključak}
% ------------------------------------------------------------------------------
% \pangrami

% \pangrami

% ------------------------------------------------------------------------------
% Literatura
% ------------------------------------------------------------------------------
\literatura

% ==============================================================================
% Završni deo teze i prilozi
\backmatter
% ==============================================================================

% ------------------------------------------------------------------------------
% Biografija kandidata
\begin{biografija}
  \textbf{Vuk Stefanović Karadžić} (\emph{Tršić,
    26. oktobar/6. novembar 1787. — Beč, 7. februar 1864.}) bio je
  srpski filolog, reformator srpskog jezika, sakupljač narodnih
  umotvorina i pisac prvog rečnika srpskog jezika.  Vuk je
  najznačajnija ličnost srpske književnosti prve polovine XIX
  veka. Stekao je i nekoliko počasnih mastera.  Učestvovao je u
  Prvom srpskom ustanku kao pisar i činovnik u Negotinskoj krajini, a
  nakon sloma ustanka preselio se u Beč, 1813. godine. Tu je upoznao
  Jerneja Kopitara, cenzora slovenskih knjiga, na čiji je podsticaj
  krenuo u prikupljanje srpskih narodnih pesama, reformu ćirilice i
  borbu za uvođenje narodnog jezika u srpsku književnost. Vukovim
  reformama u srpski jezik je uveden fonetski pravopis, a srpski jezik
  je potisnuo slavenosrpski jezik koji je u to vreme bio jezik
  obrazovanih ljudi. Tako se kao najvažnije godine Vukove reforme
  ističu 1818., 1836., 1839., 1847. i 1852.
\end{biografija}
% ------------------------------------------------------------------------------

\end{document}
